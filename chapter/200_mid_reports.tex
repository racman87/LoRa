\chapter{Rapport intermédiaire: 27 août - 18 septembre}
Ce chapitre présente les différentes étapes effectuées lors du déroulement de ce travail entre le 27 août et le 18 septembre 2018 (durée: 3 semaines).

\section{Cahier des charges}
\todo{valider.}

\subsection{Objectifs et tâches à réaliser}
\begin{enumerate}
	\item Etude théorique des réseaux de neuronnes convolutifs
	\item Etude de l'état de l'art de la reconnaissance de piétons
	\item Etude de l'état de l'art de l'utilisation d'images thermiques pour l'entrainement et l'inférence d'un réseau de neuronnes convolutif
	\item Mise en place et tests des exemples du framework Bonseyes\footnote{https://www.bonseyes.com/}
	\item Modification du framework Bonseyes pour permettre l'entrainement d'un réseau de neuronnes avec les images thermiques d'un dataset existant
	\item Evaluation des possibilité et modification du framework Bonseyes pour permettre l'entrainement d'un réseau de neuronnes avec les images thermiques et visibles d'un dataset existant
	\item Eventuellement: mise en place de l'inférence du modèle entrainé pour une reconnaissance en temps réel


\end{enumerate}

\subsection{Délivrables}
\begin{itemize}
	\item Les différentes études précitées
	\item Marches à suivre et résultats de la mise en place des exemples Bonseyes (?)
	\item Version modifiée du framework Bonseyes avec support de l'entrainement utilisant un dataset d'images thermiques et éventuellement visibles
	\item Eventuellement: plateforme de reconnaissance en temps réel
\end{itemize}

\subsection{Délais}
\begin{itemize}
	\item La remise du rapport doit être faite au plus tard le vendredi 8 février 2019.
	\item La défense se déroulera entre le 4 et le 15 mars 2019.
	\item Les tâches obligatoires doivent être finies avant le 1\ier{} juin.
	\item les tâches optionnelles devraient être commencées au plus tard le 15 janvier 2019, afin d'avoir au moins 15 jours pour proposer et tester un concept suffisant.
\end{itemize}


\section{Etat de l'art}
\todo{faire un état de l'art sur la reconnaissance multispectrale de piétons}


\section{Datasets disponibles}
Afin de procéder à l'entrainement du modèle, il est nécessaire de posséder un dataset avec des images aussi proches que possible de celles qui seront fournies lors de l'inférence.

Bien que ce projet porte sur une reconnaissance "multispectrale", il peut être intéressant de tester l'entrainement avec des datasets d'images d'un seul type: thermique ou visible.
En effet, cela peut servir à prendre en main le framework Bonseyes plus facilement, sans avoir à se préoccuper de la fusion des images dès le début.
Cela peut aussi permettre d'entrainer un modèle spécifiquement pour les images soit visibles, soit thermiques en fuisionnant les résultats à la fin.

Pour cela, 4 types de datasets seront pris en considération:
\begin{itemize}
	\item images visibles uniquement,
	\item images thermiques uniquement,
	\item images thermiques et images visibles avec le même cadrage et sans parallaxe,
	\item images thermiques et images visibles avec cadrage différent.
\end{itemize}

Comme ce projet porte sur la reconnaissance des piétons, les datasets utilisés pour l'entrainement devraient posséder les plus de piétons possibles. Dans la pratique, l'entrainement sera fait pour les classes labellisées pour chaque dataset. Il n'y aura donc pas forcément que la reconnaissance des piétons qui sera entrainée, mais potentiellement aussi d'autres classes, telles que les voitures et les vélos.

\subsection{FLIR}
FLIR Systems, un grand fabricant de capteurs et caméras infrarouge a mis a disposition, dès juillet 2018, un dataset contenant des images thermiques ainsi que visbles\footnote{https://www.flir.com/oem/adas/adas-dataset-form/}.
\todo{plus de détails sur l'acquisition, dimensions,...}
Les spécifications de ce dataset sont présentées à la table \ref{tab:dataset-flir}
\begin{table}[h!]
	\begin{center}
		\caption{Caractéristiques du dataset fourni par FLIR}
		\label{tab:dataset-flir}
		\begin{tabular}{p{4.5cm}|p{8cm}}
%			\textbf{Value 1} & \textbf{Value 2}\\ % <-- added & and content for each column
%			\hline
			Description & Images visibles (RGB) et thermiques synchronisées.\\
			& 2 caméras distantes d'environ 5cm pour minimiser la parallaxe\\
			Nombre d'images & > 14'000 provenant de segments de vidéos\\
			Labels & Piétons, voitures, vélos, chiens, autres véhicules\\
			Nombre d'annotations: & 10'228 dont 9214 avec encadrement (bounding box)\\
			Nombre de: piétons & 21'965 \\
			Nombre de voitures & 12'013\\
			Nombre de vélos & 1205\\
			Nombre de chiens & 0\\
			Nombre d'autres véhicules & 540 \\
%			1 & 1110.1 \\ % <--
%			2 & 10.1\\ % <--
%			3 & 23.113231\\ % <--
		\end{tabular}
	\end{center}
\end{table}

Ce dataset peut être téléchargé gratuitement après s'être enregistré sur le site de FLIR.

\subsection{Kaist}

KAIST\footnote{Korea Advanced Institute of Science and Technology} a développé un dataset particulièrement intéressant, comportant des images visibles ainsi que des images thermiques.

Ce dataset est disponible en ligne\footnote{http://rcv.kaist.ac.kr/multispectral-pedestrian/}, gratuitement, après enregistrement.
\todo{lire et résumer CVPR15}
Les spécifications de ce dataset sont:
\begin{itemize}
	\item 95 328 paires d'images thermiques et visibles, alignées, sans parallaxe,
	\item 103 128 annotations de piétons, avec bounding box,
	\item 1182 piétons différents.
\end{itemize}

\begin{table}[h!]
	\begin{center}
		\caption{Caractéristiques du dataset fourni par KAIST - comparaison des deux caméras}
		\label{tab:dataset-kaist2}
		\begin{tabular}{p{4.5cm}|p{5cm}|p{5cm}}
			\textbf{Description} & \textbf{Caméra visible (RGB)} & \textbf{Caméra thermique}\\
			\hline
			Modèle de caméra & PointGrey Flea3 & FLIR A35\\
			Dimension des images & 640x480 & 320x256\\
			Champ de vue & 103.6\degres vertical & 39\degres vertical\\
		\end{tabular}
	\end{center}
\end{table}

\section{Modèles disponibles?}


\section{Deep learning}


\section{CNN}


\section{Bonseyes}
Bonseyes est une plateforme d'intelligence artificelle, développé dnas le cadre d'un projet Horizon 2020.\todo{détailler!}
Le but de Bonseyes est de mettre à disposition des utilisateurs une plateforme modulable pour l'utilisation du deep-leanring.
Cette plateforme, en développement actuellement devrait permettre:
\begin{itemize}
	\item Utilisation de différents frameworks tels que Tensorflow, Caffe, Theano\dots
	\item Utilisation de plugins pour effectuer les calculs sur DSP, FPGA, GPU ou CPU.
\end{itemize}


\subsection{Prérequis et dépendances}
Bonseyes, grâce à un concept de workflows utilisant des conteneurs Docker permet une installation simple.

Les différents éléments requis sont:
\begin{itemize}
	\item Un PC possédant une carte graphique Nvidia,
	\item Une distribution GNU/Linux,
	\item Docker
	\item Nvidia-docker
	\item Drivers Nvidia
	\item git
\end{itemize}	

\subsection{Architecture}

\todo{parler de l'architecture, des tools, workflows,...}

\subsection{Installation des dépendances et de be-admin}
Afin de pouvoir utiliser les différents workflows existant et en développer de nouveau, il est nécessaire d'installer les dépendances de base de Bonseyes ainsi que d'installer l'outil 'administration des workflows (be-admin).

Dans cet exemple, ces instructions s'appliquent à une machine fonctionnement sur Ubuntu 16.04 LTS.

Installation des drivers nvidia
\begin{lstlisting}
sudo add-apt-repository ppa:graphics-drivers/ppa
sudo apt-get update
sudo apt-get install nvidia-384
\end{lstlisting}

Installation de git, docker et nvidia-docker
\begin{lstlisting}
sudo apt-get update
sudo apt-get install git docker nvidia-docker
\end{lstlisting}




\subsection{Execution du workflow MNIST}

Le jeu de données MNIST\footnote{Modified National Institute of Standards and Technology} est une compilation de 70 000 images de chiffres manuscrits. Ce dataset convient particulièrement bien pour différents tests, tels que la vérification du bon fonctionnement de Bonseyes, car c'est un dataset assez simple, contenant des images de faibles dimensions\footnote{28x28 pixels} avec un nombre restreint de classes\footnote{10 classes représentant les chiffres de 0 à 9}.

Afin de pouvoir procéder à ces tests, il existe dans le projet Bonseyes, un sous-projet appelé "wp3-project-mnist" qui est l'implémentation de l'entrainement et du test d'un MLP\footnote{Multilayer Perceptron} utilisant le dataset MNIST.



Les instructions pour la mise en place et l'exécution de ce workflow se trouvent à l'adresse \url{https://bitbucket.org/bonseyes/wp3-project-mnist/src/master/}

\begin{lstlisting}
{
	"application_config" : {
		"run_opts": {
			"volumes": {
				"nvidia_driver_384.130": { "bind": "/usr/local/nvidia", "mode": "ro"}
			},
			"devices": [
			"/dev/nvidiactl",  "/dev/nvidia-uvm-tools", "/dev/nvidia-uvm", "/dev/nvidia0"
			]
		}
	}
}
\end{lstlisting}

\subsection{Execution du workflow Object Detection}

\section{Utilisation du dataset Kaist dans le framewrok object detection}
