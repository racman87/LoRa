\chapter{Rapport intérmédiaire : 17.09.2018 au 12.10.2018}

Ce premier résumé a pour but de poser le projet et d'étudier l'état de l'art. Depuis les lectures concernant ce qu'il existe en machin learning pour le positionnement indoor, il est nécessaire de faire un résumé afin de choisir le meilleur algorithme afin d'améliorer le positionnement indoor à l'aide de la technologie LoRa et le mode ranging.

\section{Cahier des charges}
\subsection{Introduction}
Les systèmes de localisation basés sur GPS souffrent de la détérioration de la précision et sont presque indisponibles dans les environnements intérieurs. Pour les environnements intérieurs, de nombreuses technologies de systèmes de positionnement ont été conçues sur la base de la vision, de la détection infrarouge ou ultrasonore, des champs magnétiques de la terre, des accéléromètres / gyromètres, des balises BLE ou de la communication WiFi. Chacune de ces technologies existantes a des coûts, une précision et un compromis maximum en matière de couverture, mais un service de localisation intérieur générique reste difficile à obtenir.

\subsection{But du projet}
S'appuyant sur les capacités étendues des nouveaux circuits intégrés LoRa, ce projet développera et déploiera un système de localisation capable d'améliorer la précision de la position atteinte par les systèmes de localisation basés sur LoRa existants reposant sur des mécanismes TDOA ou de télémétrie. À cette fin, une exploration et une comparaison des différentes techniques "machine learning/deep learning" pour le positionnement basé sur le "fingerprinting" seront effectuées.


\subsection{Objectifs et tâches à réaliser}
\begin{enumerate}
	\item Etudier le cahier des charges
	\item Etudier l’état de l’art des techniques à utiliser dans le cadre du projet, en particulier les systèmes de localisation indoor basés sur des techniques d’apprentissage, et réunir une documentation (env. 20% effort)
	\item Etablir un planning pour l’ensemble du projet.
	\item Définir un plan des tests à effectuer.
	\item Définir les procédures de test
	\item Définir le setup pour la collecte de données de localisation
	\item Prise en main de l’environnement de développement pour les phases de training et du test de la technique d’apprentissage retenue (e.g., PyTorch).
	\item Implémentation de la solution ML retenue.
	\item Tester le système selon le protocole préétabli.
	\item Faire des propositions pour améliorer les performances de l’algorithme et, si possible, les implémenter.
	\item Rédiger le rapport et documenter l’ensemble du projet.
\end{enumerate}


\section{Résumé du document 00}
\subsection{Pro/con}

\section{Résumé du document 01}
\subsection{Pro/con}

\section{Résumé du document 02}
\subsection{Pro/con}

\section{Résumé du document 03}
\subsection{Pro/con}