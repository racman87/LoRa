\chapter{Définition d'une base réflexion pour un protocole sécurisé sur un bus CAN}
\blindtext

\lstset{
	language=C,
	showstringspaces=false,
	basicstyle=\ttfamily\small,
	commentstyle=\itshape,
	morecomment=[s]{"""}{"""},
}
\lstinputlisting[caption={Exemple d'écriture en boucle d'une trame CAN en utilisant le framework Python "pyvit" - code source}, label={lst:sample}]{source_code/sample.c}

\section{Buts}
\blindtext

\begin{itemize}
	\item La taille des données pouvant être transmises doit être plus grande que 8 bytes \footnote{Taille maximum du champ de données sur le bus CAN 2.0A/B} afin de pouvoir contenir une signature, car habituellement une signature a une taille d'au moins une centaine de bits au minimum.
	\item Chaque émetteur de données doit pouvoir signer ses messages avec une clé privée.
	\item Chaque récepteur de données doit pouvoir vérifier les messages reçus avec une clé publique qui doit être autorisée par un élément central auquel il a été décidé de faire confiance.
	\item Les clés publiques doivent être stockées chez un gestionnaire de clés publiques, afin de pouvoir modifier les périphériques autorisés à envoyer des messages aisément.
\end{itemize}

\subsection{Paquet CAN standard}
Lors de l'utilisation de la surcouche CanSecure, chaque paquet de 8 bytes envoyé sur le bus CAN comporte les données suivantes dans son champ de données (voir figure \ref{fig:pack-can}):
\begin{itemize}
	\item byte 0: frame ID
	\item byte 1: packet number
	\item bytes 2-7: data (6 bytes)
\end{itemize}

Le champ "frame ID" permet d'identifier un paquet CanSecure.\\
Le champ "packet number" permet d'identifier l'emplacement du paquet fragmenté dans le paquet CanSecure.\\
Le champ "data" permet d'envoyer 6 bytes de données utiles du paquet CanSecure.

\myfig{CanSecure/pack-can-2.png}{0.8}{Paquet CAN standard}{fig:pack-can}
\FloatBarrier

\paragraph{Dans le meilleur des cas:}
Lors de l'envoi de 1404 bytes de données, avec le périphérique recevant les données ayant déjà la clé publique en cache, il faut transmettre $\lceil\frac{1404+132}{6}\rceil=256$ paquets CAN.\\
Chaque paquet CAN a une taille de 108 bits (CAN 2.0A, 64 bits de données, en omettant le bit stuffing).\\
Cela implique donc de transférer $256*108 = 27648$bits $= 3456$bytes.\\
Ce qui donne donc: $overhead = \frac{3456}{1404} = 2.46$.


C'est la raison pour laquelle, dans le protocole TLS, les deux parties (client et serveur) commencent tout d'abord par une phase de "handshake" permettant au client et au serveur de s'échanger des clés publiques et des certificats, puis un secret partagé est généré, permettant un fonctionnement à base de cryptographie à clés privées, voir \cite{rfc5246}, même si chaque paquet comporte néanmoins très souvent une signature à base de clé publique.
